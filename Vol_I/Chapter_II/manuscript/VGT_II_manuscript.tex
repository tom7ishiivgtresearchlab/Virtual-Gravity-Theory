\documentclass[aps,prd,onecolumn,nofootinbib,superscriptaddress]{revtex4-2}

\usepackage{graphicx}
\usepackage{amsmath,amssymb,bm}
\usepackage{hyperref}
\usepackage{booktabs}
\usepackage{multirow}
\usepackage{float}

\graphicspath{{./}}
\hypersetup{colorlinks=true,linkcolor=blue,citecolor=blue,urlcolor=blue}

\begin{document}

\title{Virtual Gravity Theory (VGT) Part II: Theoretical Consistency, Scale-Dependent Dynamics, and Cosmological Reconstruction (Phases 4--11)}

\author{Tsutomu Ishii}
\affiliation{Independent Researcher, VGT Research Lab, Japan}
\email{vgt.researchlab@gmail.com}
\thanks{ORCID: 0009-0001-3019-3929 \quad GitHub: \url{https://github.com/tom7ishiivgtresearchlab}}
\date{\today}

\begin{abstract}
We present the second stage (Phases 4--11) of the Virtual Gravity Theory (VGT) program, extending the empirical foundation in Part~I (Phases~1--3; Ishii 2025, \textit{submitted to arXiv}). 
VGT replaces the cosmological constant and cold dark matter paradigm with a scale-dependent coupling $G_{\rm eff}(z)=G_N[1+\alpha\ln(1+z)]$, motivated by renormalization-group flows in effective field theory where couplings run slowly with a characteristic scale $\sim H(z)$. 
Using 57 measurements (cosmic chronometers + BAO + SN\,Ia), we find $|\Delta G/G|<6.5\times10^{-5}$ over $z\!\lesssim\!2.3$ and $\sigma_8(z=1)=0.78\pm0.03$, consistent with Planck + BOSS within 2\%. 
Phases 10--11 provide falsifiable predictions: Euclid weak lensing should measure $\sigma_8(z=1)=0.78\pm0.03$ and a $1.5\sigma$ high-$z$ structure excess vs.\ $\Lambda$CDM. 
A detection of $\sigma_8(z=1)<0.75$ would exclude the present VGT parameterization at $>3\sigma$.
\end{abstract}

\maketitle

\section{Introduction}
The $\Lambda$CDM cosmology explains CMB anisotropies, large-scale structure, and SN\,Ia luminosities \cite{Planck2018,DESI2024,Riess2022}. 
Outstanding issues include the $10^{120}$ vacuum-energy gap \cite{Weinberg1989}, null dark-matter detections \cite{Bertone2005,Clowe2006}, and persistent $H_0/\sigma_8$ tensions \cite{Riess2022,DES2022,DiValentino2021}. 
Alternatives such as $f(R)$ gravity \cite{DeFelice2010}, scalar--tensor theories \cite{Fujii2003}, and coupled dark energy \cite{Wetterich1995} often add parameters or fields.

\textbf{VGT} follows a minimal route: a scale-dependent $G_{\rm eff}(z)$ absorbs apparent dark-sector effects while recovering GR in appropriate limits. 
Part~I established observational consistency for a slowly varying $G_{\rm eff}$; here (Phases 4--11) we develop theoretical consistency, reconstruct the effective field, and derive survey-level predictions.

\section{Theoretical Framework (Phases 4--7)}
\subsection{Scale-dependent coupling (Phase 4)}
We extend the Einstein--Hilbert action with a virtual field $\Phi$:
\begin{equation}
S=\frac{1}{16\pi G_N}\!\int d^4x\,\sqrt{-g}\,\bigl[R-2\Lambda+f(\Phi,\partial_\mu\Phi)\bigr]+S_m.
\end{equation}
A first-order expansion yields $G_{\rm eff}(z)=G_N[1+\alpha(z)]$ with
$\alpha(z)=\alpha_0+\alpha_1\ln(1+z)$, a logarithmic running natural in RG flows \cite{Burgess2004,Wetterich1988}.
In flat FLRW,
\begin{equation}
H^2(z)=\frac{8\pi G_{\rm eff}(z)}{3}\rho_m(z)+\frac{\Lambda}{3}.
\end{equation}

\subsection{Vacuum interaction \& early-universe consistency (Phases 5--7)}
Vacuum--curvature feedback is encoded via
\begin{equation}
V_{\rm eff}(k,z)=V_0(k)[1+\mu(k,z)],
\end{equation}
with $\mu\!\to\!0$ at small scales and $\mu>0$ on cosmic scales.
Halo statistics follow a Press--Schechter form with growth modified by $\alpha(z)$; at $z_{\rm rec}\!\approx\!1090$, $|\alpha|\!\lesssim\!10^{-4}$, so $\Delta r_s/r_s$ is negligible.

\section{Reconstruction and Observational Analysis (Phases 8--9)}
\subsection{Field reconstruction (Phase 8)}
From Raychaudhuri,
\begin{equation}
\frac{d\Phi}{dz}=\frac{1}{H(1+z)}\!\left[\frac{dH}{dz}-\frac{3}{2}(1+w_m)H\right],
\end{equation}
yielding a slowly varying $\Phi(z)$.

\subsection{Joint constraints (Phase 9)}
We fit CC+BAO+SN\,Ia (57 points) with flat priors on $(\alpha_0,\alpha_1,\Omega_m,H_0)$ via MCMC.
Table~\ref{tab:params} lists best fits.

\begin{table}[H]
\centering
\caption{Best-fit parameters (1$\sigma$) and $\Lambda$CDM references.}
\label{tab:params}
\begin{tabular}{lcc}
\toprule
Parameter & VGT best-fit & $\Lambda$CDM reference \\
\midrule
$H_0$ [km s$^{-1}$ Mpc$^{-1}$] & $69.4 \pm 0.7$ & $68.6 \pm 0.5$ \\
$\Omega_m$ & $0.307 \pm 0.012$ & $0.315 \pm 0.007$ \\
$\alpha_0$ & $(3.8 \pm 0.9)\!\times\!10^{-5}$ & $0$ \\
$\mu_0$ & $0.06 \pm 0.02$ & $0$ \\
$\chi^2_{\rm red}$ & $1.08$ & $1.05$ \\
\bottomrule
\end{tabular}
\end{table}

\begin{figure}[H]
\centering
\includegraphics[width=0.80\linewidth]{../figures/VGT_II_fig1_GeffEvolution.png}
\caption{Reconstructed $G_{\rm eff}(z)/G_N$ evolution with 68\% confidence regions from CC+BAO+SN\,Ia joint analysis (Phase 8--9). Blue: VGT posterior; dashed: $\Lambda$CDM baseline ($\alpha=0$).}
\label{fig:geff}
\end{figure}

\begin{figure}[H]
\centering
\includegraphics[width=0.85\linewidth]{../figures/VGT_II_fig2_BAODistance.png}
\caption{BAO distance-scale comparison: SDSS/BOSS DR12 \cite{Alam2017} (red points with 1$\sigma$ error bars) vs.\ VGT prediction (blue solid) and $\Lambda$CDM baseline (black dashed). $\chi^2_{\rm red}=1.03$ for VGT.}
\label{fig:bao}
\end{figure}

\begin{figure}[H]
\centering
\includegraphics[width=0.80\linewidth]{../figures/VGT_II_fig3_SNIaDistMod.png}
\caption{SN\,Ia Hubble residuals: VGT (blue solid) vs.\ $\Lambda$CDM (black dashed). Binned Pantheon+ compilation \cite{Brout2022} with 1$\sigma$ error bars. Residuals $\Delta\mu < 0.05$ mag across $0.01 < z < 2.3$.}
\label{fig:snia}
\end{figure}

\section{Stability and Consistency (Phases 10--11)}
Linearized $\delta\Phi$ obeys
\begin{equation}
\ddot{\delta\Phi}+3H\dot{\delta\Phi}+M_{\rm eff}^2\delta\Phi=0,
\end{equation}
with $M_{\rm eff}^2=\partial^2 V_{\rm eff}/\partial\Phi^2>0$ for $|\mu|<0.1$ (no tachyonic mode).
The running gives 
\begin{equation}
\frac{\dot G_{\rm eff}}{G} \approx H_0\alpha_1 \lesssim 2\times10^{-13}\,{\rm yr}^{-1},
\end{equation}
consistent with LLR/pulsars (Table~\ref{tab:consistency}) and future-survey predictions (Table~\ref{tab:predictions}).

\begin{table}[H]
\centering
\caption{Independent constraints and VGT consistency checks.}
\label{tab:consistency}
\begin{tabular}{lcc}
\toprule
Constraint source & Observed limit & VGT prediction \\
\midrule
Lunar Laser Ranging & $<7\times10^{-13}$ yr$^{-1}$ & $2\times10^{-13}$ yr$^{-1}$ \\
Binary pulsars (PSR J0737) & $<1\times10^{-12}$ yr$^{-1}$ & $2\times10^{-13}$ yr$^{-1}$ \\
CMB acoustic peaks & $\Delta r_s/r_s < 0.3\%$ & $< 0.1\%$ \\
BBN (D/H ratio) & consistent & consistent \\
\bottomrule
\end{tabular}
\end{table}

\begin{table}[H]
\centering
\caption{VGT predictions for next-generation surveys (2025--2030).}
\label{tab:predictions}
\begin{tabular}{lccc}
\toprule
Observable & VGT prediction & $\Lambda$CDM baseline & Survey \\
\midrule
$\sigma_8(z=1)$ & $0.78 \pm 0.03$ & $0.75 \pm 0.02$ & Euclid WL \\
$H(z=2)$ [km s$^{-1}$ Mpc$^{-1}$] & $225 \pm 8$ & $220 \pm 5$ & DESI BAO \\
High-$z$ structure ($z>7$) & $+1.5\sigma$ excess & baseline & JWST \\
Galaxy--void correlation & $-0.8\%$ shift & baseline & LSST \\
\bottomrule
\end{tabular}
\end{table}

\section{Discussion}
\textbf{Central result:} a scale-dependent $G_{\rm eff}(z)$ unifies acceleration and structure growth without new particles, matching 57 measurements at $\sim2\%$ while satisfying Solar-System bounds. 
Limitations include the lack of a quantum-gravity derivation of $f(\Phi)$ and early-universe initial conditions. 
Comparisons with $f(R)$ gravity reveal VGT's advantage: $f(R)$ typically requires $|f_R| \sim 10^{-6}$ fine-tuning, whereas VGT's $\alpha_0 \sim 10^{-5}$ emerges naturally from RG running.
Future work (Phases 12--15) will treat quantum-vacuum feedback, lab-scale tests (torsion balance, atom interferometry), and full Euclid/DESI likelihood analyses.

\section{Conclusion}
Phases 4--11 complete VGT's theoretical and observational core. 
The framework reproduces $\Lambda$CDM in the appropriate limits yet predicts percent-level deviations testable by Euclid (2025--2027) and Rubin Observatory (LSST, 2025--2030).
A null detection of VGT signatures would constrain $|\alpha_1| < 10^{-6}$, refining our understanding of scale-dependent gravity.

\begin{acknowledgments}
The author thanks collaborative AI systems (Claude, Anthropic) for editorial feedback and acknowledges public cosmological datasets (Planck, SDSS, DESI, Pantheon+) and open-source MCMC tools (emcee, GetDist).
Code and data are available at \url{https://github.com/tom7ishiivgtresearchlab/VGT-Phase4-11}.
\end{acknowledgments}

\bibliographystyle{apsrev4-2}
\bibliography{references}

\end{document}
